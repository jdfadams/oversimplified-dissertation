\documentclass{beamer}
\usepackage{amsmath,amsthm}
\usepackage{amssymb}
\usepackage[english]{babel}
\usepackage{latexsym}
\usepackage{amsfonts}
\usepackage{graphicx}
\usepackage{float}
\usepackage{graphics}
\usepackage{epsfig}
\usepackage{url}
% \usetheme{Madrid}
\usetheme{Boadilla}
% \usecolortheme{Boadilla}
\usepackage{multirow}

\mode<presentation>

\title{Renormalization and rigidity}

\author{Joseph Adams}
\date{\today}

\begin{document}
\begin{frame}

\titlepage

\end{frame}

\begin{frame}
    \frametitle{Overview}
    \tableofcontents
\end{frame}

\begin{frame}{The big picture}
\begin{itemize}
    \item Define renormalization of quadratic dynamical systems
    \item Give sufficient conditions for the existence of \emph{a priori} bounds
    \item Prove that these conditions imply combinatorial rigidity
    \item To non-mathematicians: Sorry... This is a math talk.
    \item To mathematicians: Sorry... I'll skip all the details.
\end{itemize}
\end{frame}

\section{Background}
\begin{frame}{Complex numbers}

\begin{definition}
A \emph{complex number} has the form $a+bi$, where $a$ and $b$ are real numbers, and $i=\sqrt{-1}$. $\mathbb{C}$ denotes the set of all complex numbers.
\end{definition}

\begin{itemize}
    \item Addition: $(a+bi)+(c+di)=(a+c)+(b+d)i$
    \item Multiplication: $(a+bi)\cdot(c+di)=(ac-bd)+(ad+bc)i$
    \item We can visualize the complex numbers as a 2-dimensional plane.
\end{itemize}

% \begin{itemize}
% \item First Sentence.
% \item<2-> ``Appears''
% \item<2->  Appears. Brueckner(2000)
% \end{itemize}
\end{frame}

\begin{frame}{Complex functions}
    \begin{itemize}
        \item Elementary calculus considers functions $f:\mathbb{R}\rightarrow\mathbb{R}$.
        \item Elementary calculus generalizes to functions $f:\mathbb{C}\rightarrow\mathbb{C}$.
        \item A differentiable complex function $f$ has a derivative $f'$. Zeros of the derivative are \emph{critical points}
        \item Complex functions can be iterated: $f^2(z)=f(f(z))$. The \emph{$n$-th iterate} of $f$ is $f^n(z)=\underbrace{f(...(f}_{n\text{ times}}(z))$.
    \end{itemize}
    \begin{example}
        A \emph{complex quadratic} has the form $f(z)=z^2+c$, where $c$ is a complex number.
        \begin{itemize}
            \item $f'(z)=2z$, so $0$ is the critical point.
            \item $f^2(z)=(z^2+c)^2+c$.
        \end{itemize}
    \end{example}
\end{frame}

\section{Dynamics}

\begin{frame}{The Fatou set}
Consider a complex quadratic $f_c(z)=z^2+c$.
\begin{definition}
The \emph{Fatou set} $K_c$ of $f_c$ is the set of $z$ such that $f_c^n(z)\not\rightarrow\infty$ as $n\rightarrow\infty$.
\end{definition}
\begin{definition}
The \emph{Julia set} is the boundary of $K_c$.
\end{definition}
For historical reasons, we'll sometimes call $K_c$ the Julia set.
\end{frame}

\begin{frame}{The Julia set can be connected}
\centering
$c=-0.1226+0.7449i$

\includegraphics[scale=0.25]{k_rabbit.png}
\end{frame}

\begin{frame}{The Julia set can be disconnected}
\centering
$c=-0.781+0.234j$

\includegraphics[scale=0.25]{k_cantor.png}
\end{frame}

\begin{frame}{Parameter space for quadratic dynamics}
\begin{theorem}[The fundamental dichotomy]
$K_c$ is either connected or totally disconnected.
\end{theorem}
\begin{theorem}
$K_c$ is connected if and only if $f_c^n(0)\not\rightarrow\infty$.
\end{theorem}
\begin{definition}
The \emph{Mandelbrot set} $M$ is the set of $c$ such that $K_c$ is connected.
\end{definition}
\end{frame}

\begin{frame}{The Mandelbrot set}
\centering
\includegraphics[scale=0.25]{m_default.png}

Can you see the little copies of the Mandelbrot set?
\end{frame}

\section{Renormalization}

\begin{frame}{Period 3: Little copy}
\centering
$c=-1.754877666246693$

\includegraphics[scale=0.2]{m_p3.png}
\includegraphics[scale=0.2]{m_default.png}
\end{frame}

\begin{frame}{Period 3: Julia set}
\centering
$c=-1.754877666246693$

\includegraphics[scale=0.25]{k_p3.png}
\end{frame}

\begin{frame}{Period 4: Little copy}
\centering
$c=-0.156520166833755+1.032247108922832i$

\includegraphics[scale=0.2]{m_default.png}
\includegraphics[scale=0.2]{m_p4.png}
\end{frame}

\begin{frame}{Period 4: Julia set}
\centering
$c=-0.156520166833755+1.032247108922832i$

\includegraphics[scale=0.25]{k_p4.png}
\end{frame}

\begin{frame}{Are there little-little copies?}

{\bf Observations}
\begin{itemize}
    \item Little copies look like the Mandelbrot set.
    \item We've seen little copies for periods 3 and 4.
\end{itemize}
{\bf Question}

Is there a little-little copy in the period 4 little copy corresponding to the period 3 little copy in the Mandelbrot set?
\end{frame}

\begin{frame}{Period 12: Little copy}
\centering
$c=-0.167349208205021+1.041178661132973i$

\includegraphics[scale=0.2]{m_p12.png}
\includegraphics[scale=0.2]{m_p4.png}
\end{frame}

\begin{frame}{The period 12 LC corresponds to the period 3 LC}
\centering
\begin{tabular}{c c c}
    \begin{tabular}{c}\includegraphics[scale=0.12]{m_p12.png}\end{tabular} &
    \begin{tabular}{c}is to\end{tabular} &
    \begin{tabular}{c}\includegraphics[scale=0.12]{m_p4.png}\end{tabular}
    \\
    & as &
    \\
    \begin{tabular}{c}\includegraphics[scale=0.12]{m_p3.png}\end{tabular} &
    \begin{tabular}{c}is to\end{tabular} &
    \begin{tabular}{c}\includegraphics[scale=0.12]{m_default.png}\end{tabular}
\end{tabular}
\end{frame}

\begin{frame}{Period 12: Julia set}
\centering
$c=-0.167349208205021+1.041178661132973i$

\includegraphics[scale=0.25]{k_p12.png}

It looks like a little period 3 Julia set is embedded in the period 4 Julia set.
\end{frame}

\begin{frame}{Renormalization 1: Zoom in to the little Julia set}
\centering
Zoom in to the center of the period 12 Julia set.

\includegraphics[scale=0.2]{k_p12.png}
\includegraphics[scale=0.2]{k_p12_zoom.png}
\end{frame}

\begin{frame}{Renormalization 2: Straighten out the little Julia set}
\centering
\includegraphics[scale=0.2]{k_p12_zoom.png}
\includegraphics[scale=0.2]{k_p3.png}

We obtain the period 3 Julia set.
\end{frame}

\begin{frame}{Combinatorial model: Forget about the little Julia set}
\centering
Blur your eyes so that each little period 3 Julia set just looks like a blob.

\includegraphics[scale=0.2]{k_p12.png}
\includegraphics[scale=0.2]{k_p4.png}

We obtain the period 4 Julia set.
\end{frame}

\begin{frame}{Summary of renormalization}
Let $f$ be a \emph{renormalizable} quadratic.
\begin{definition}
The \emph{renormalization} $Rf$ of $f$ is the quadratic obtained by zooming in to a little embedded Julia set.
\end{definition}
Strictly speaking, this renormalization scheme is called \emph{primitive} renormalization.
\begin{definition}
The \emph{combinatorial model} $Hf$ of $f$ is the shape/dynamics of the big Julia set. We can think of it as the quadratic obtained by replacing each little Julia set with a blob/disk.
\end{definition}
In mathematical literature, the combinatorial model is actually the \emph{Hubbard tree}.
\end{frame}

\begin{frame}{The big picture}

\end{frame}

\begin{frame}{Zooming in}

\end{frame}

\begin{frame}{Little copies}
\centering
\includegraphics[scale=0.7]{little_copy}
\end{frame}

\begin{frame}{Zooming in}

\end{frame}

\section{Frame title}

 \begin{frame}
       \frametitle{Outline}
       \tableofcontents
\end{frame}
\begin{frame}

\frametitle{Frame title}


\begin{itemize}
\item<1>[1.] ``Appears and dissapears.'' Burchell et al (1998)
\item<1>[2.] Appears and dissapears.
\item<2->[3.] Reappears
\begin{itemize}
\item<2-> Appears;
\item<2-> Appears
\item<2-> Appears
\item<2-> Appears
\end{itemize}
\end{itemize}

\end{frame}





\begin{frame}
\frametitle{Descriptive statistics}

Add an image or table.
\begin{center}
\includegraphics[scale=0.85]{imagen1.png}%
\end{center}
\end{frame}



\begin{frame}
\frametitle{Model}

\begin{equation}
Y_i=\beta_0 + \beta_1 X_i+\beta_2 X_{2i} + \beta_3 Other_i + \gamma S_i+u_i
\end{equation}



\begin{columns}
\column{.5\textwidth}
\begin{block}{Y: Externality}
\begin{itemize}
\item Var1.
\item Var2.
\item Var3.
\item Var4.
\end{itemize}
 \end{block}
\begin{block}{Var7}
\end{block}

\column{.5\textwidth}
\begin{block}{Var8: Socioeconomic characteristics}
\begin{itemize}
\item Var8.1
\item Var8.2.
\item Var8.3.
\item Median income.
\item Median home value.
\end{itemize}
\end{block}
\begin{block}{S: State dummies}
\end{block}
\end{columns}
\end{frame}


\section{Results}

 \begin{frame}
       \frametitle{Outline}
       \tableofcontents
\end{frame}

\begin{frame}
\frametitle{Frame title}


\begin{center}
\includegraphics[scale=0.7]{imagen1.png}%
\end{center}
\end{frame}

\begin{frame}
\frametitle{ReSULT}


\begin{center}
\includegraphics[scale=0.75]{imagen2.png}%
\end{center}
\end{frame}

\begin{frame}
\frametitle{Result}


\begin{table}[!ht]
    \centering
    \begin{tabular}{|l|l|l|}
    \hline
        Name & Turn & Height \\ \hline
        Juan & 1 & 1.9 \\ \hline
        Jose & 2 & 1.7 \\ \hline
        Michael  & 3 & 1.95 \\ \hline
    \end{tabular}
\end{table}
\end{frame}


\section{Discussion}



\begin{frame}
\frametitle{ Discussion}

\begin{itemize}
\item[1.] Comment1
\item[2.] Comment2.
\item[3.] Comment3.
\item[4.] Comment4.
 \end{itemize}

\end{frame}








\end{document}
